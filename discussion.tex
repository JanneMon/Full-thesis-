\chapter{Discussion and future work}
\label{sec:discussion}

In this chapter the results will be discussed in more detail for the purpose of proposing any issues and aspects that needs to be addressed in future work for obtaining further knowledge and improvement of the field. 

\section{Grid considerations}

The grid of tracks constructed for this project proves sufficient for providing an estimate on the range in the evolutionary stage for 44 Tau and HD 187547. It stretches from a mass of 1.5 to 2.2 \msun to cover up most of the instability strip in the parameters space. However, as $\delta$ Sct stars have been observed even outside of the instability strip, it is important to include that possibility of more extreme parameter space possibilities. Therefore, further analysis in the future should include a more detailed evaluation of the outer grid bounds to analyze the probability of the best fit models in those ranges. As th exact position of the instability strip is still a big discussion topic in the field of asteroseismology of $\Delta$ Sct stars \citep{murphy2019gaia}, a larger grid including masses up to 2.5 \msun should be included in the grid to ensure full coverage. 

Since $\alpha_{mlt}$ is a purely empirical parameter and therefore not directly applicable in analytical solutions, there are not many guidelines as how to the range in the grid should be chosen. Here, smaller values were initially chosen due to two reasons: Firstly, the values for the mixing length parameter in \citep{lenz2010delta} is as low as $\alpha_{mlt}$ = 0. Hence, the values in this project must be correspondingly small in order to compare results more easily. However, this argument becomes faulty as soon as the implementations of $\alpha_{mlt}$ is not the same in both stellar structure and evolution codes, meaning that the same number has different interpretations. Although, secondly, one might still argue that such small values still is a good estimate for stars with very small convective envelopes, since the convection is not efficient. This is very much still up to discussion, since convective efficiency and how it affects the excited frequencies, despite efforts (references), still lacks understanding. For this project, the effects of different mixing length throughout the HRD is shown on \figref{mlt}, and as earlier discussed in \secref{compute} it primarily has affects the star in the late stage of the subgiant phase. 

The choice of a consistently small $\alpha_{mlt}$ could potentially cause a bias in the best 5\% as $\alpha_{mlt}$ correlates with the structure of the star (and thereby the frequencies). For future work, it would therefore be necessary to extend the $\alpha_{mlt}$ to values as high as the standard solar value of 1.8 in \texttt{MESA}. This ensures the exclusion of unconscious bias from the grid  choice. 
 
As mentioned in \secref{sec:chi44}, the uncertainties of the frequencies are very small, causing the values of the total \chis is completely controlled by the frequencies. A method taking the resolution into consideration was implemented for the purpose of weighing the frequencies with the resolution through \eqref{function}.  This did yield lower values of \chis, but not nearly enough to make the uncertainties of the frequencies the same order as the rest of the fitting parameter uncertainties. It can also be argued that frequencies should naturally be weighed higher as the observations have smaller uncertainties, and that the \logg, \teff, and \lum should be given correspondingly smaller weights due to higher uncertainties (and disagreements between spectroscopic and photometric values). However, this does cause the frequencies to weigh much more than the remaining parameters, and it would therefore be relevant to conduct an implementation of more thoroughly selected weights for all the fitting parameters. . 

Even though the frequency resolution was implemented into the \chis, it is still favorable to have as many points on a track as computationally realistic. As discussed, this is time consuming computationally, but ensures  more input parameter combinations an thereby a higher possibility for including a model with a good fit. The \texttt{target} command in \texttt{MESA} allowed for a somewhat better resolution (particularly for tracks with high $\alpha_{ov}$), but more options to increase the resolution even further should be considered in future work. This analysis was primarily focused on the evolutionary stage of the stars, but also depends strongly on the input parameters for the tracks. In this work, the step size of the tracks is bigger than that of \citet{lenz2010delta}, resulting in a larger difference in parameter space. Thus, models in between these tracks are not considered. Since creating an entire grid is time consuming, doubling it up would take very long, and could prove excessive particularly in those areas of the HRD where it would be less likely to find $\delta$ Sct pulsations. Instead, grid with smaller step sizes could be constructed around the tracks of either the best 5\% models, or those with higher likelihood. The latter is, however, a different statistical approach where each model input parameter gives a likelihood. Both ways would also allow for a more detailed evaluation of the individual model parameters, particularly the $\alpha_{mlt}$ and $\alpha_ov$. By further constructing a narrow but dense grid around the best model, it can be tested if the minimal \chis is still on the original track, or if \chis can be optimized even further.  

%This would also contain more information on not only the best stage estimate, but the parameter combination as well. The general approach used here is     

Throughout this project, the grid constructed was only calculated using GN93 element abundances and OPAL opacities. However, as \citet{lenz2010delta} showed in his asteroseismic modeling of 44 Tau,  the tracks and thereby the evolutionary stage are affected by this choice. Therefore, a further analysis of the choice is needed in order to evaluate how it affects the \chis.  


\section{Stringent statistics}

From a statistical point of view, the \chis test carried out in this work is crudely simplified for the purpose of answering the question of which evolutionary stages the stars are in. The \chis is merely a number that gives the sum of differences for the parameters, but a small \chis does not necessarily mean that that model has the highest \textbf{likelihood}. In order to compare \chis between all models in the grid, strictly speaking, the conditions should not change between the models as the individual parameters are then weighted differently. Therefore, since \eqref{function} adds a weight to the frequencies but not to the rest of the fitting parameters, the \chis sum depends on two different weighting methods. This is not a problem in itself, but it is more statistically correct to ensure consistency.  This means that comparing the same tracks for a different observed luminosity needs to be done very carefully since the uncertainties of the luminosities are different, and the \chis are thereby automatically weighted differently. Strictly speaking, the two runs for 44 Tau (with different observation parameter sets) can not be compared since they do not have the same underlying conditions. But since the frequencies dominate the run, this is not an issue (as the frequency fit is the exact same in both cases). 

In order to use the \chis as a way of determining how well a \textbf{fit} the parameters provide,  the reduced \chis should be used instead. This is defined as 

\begin{equation}
	\chi_{red}^2 = \frac{\chi^2}{K} ,
\end{equation}

\noindent where $K$ is the number of degrees of freedom. If the \chis is larger than 1, the fit is considered "bad" in the sense that the value is high compared to degrees of freedom. However, if \chis is smaller than one, it is considered an overfit. The reduced \chis is commonly used in astronomy. However, determining the number of degrees is trivial at all. For linear models it is simply given by $N-P$, where $N$ is the number of datapoints and $P$ is the number of fit parameters. For the most this is works in practice. But it is strictly speaking not true  since K can only be defined so if the basis functions are linearly independent in the sampled regime. Even though this is not the case for the modeling in this work, a reduced \chis should be considered to make a more complete evaluation of the \chis fits. 

\section{Non-adiabatic calculations}

Implementing non-adiabatic calculations of the most relevant models should be considered for future work in order to get a full discription of the frequencies. This includes information of whether modes are excited or not, and a fuller picture of the excited frequency range. This is very useful for modeling since the range can be compared to the observed frequency range. Adiabatic calculations assume adiabadicity and solves purely linear equations. However, the non-adiabatic case is non-linear and therefore more complicated and time consuming. Making non-adiabatic solutions for every model in a grid would be excessive. Therefore, as mentioned earlier in this chapter, models should be initially evaluated to choose a set of models the calculations should be conducted on. First and foremost the pre-MS models should not be considered as there is no evidence to support that this is the evolutionary stage of 44 Tau and HD 187547. If the possibility is to be considered, the \texttt{MESA} tracks should be recalculated with the main purpose of optimizing them for stellar pulsations.  Secondly, models outside the instability strip should not be included unless there is evidence that they are within reasonable fitting area. A way of doing so could be to calculate the adiabatic tracks and evaluate those (like here), and then additionally calculating the non-adiabatic values for the 5\% or 10\% best models. 

