\chapter{Introduction}

The field of astronomy is one of the oldest sciences in the world, attempting to answer any question on the universe an the celestial bodies within it. Throughout the years it has developed into several theoretical and observational branches studying every corner of the universe from the nearest objects such as the moon an the Sun to the very edges of the universe that are so far away that the human brain cannot grasp the sheer size of it. Since the early studies to present day, astronomy has developed substantially, yet many questions remain still unanswered. What does the universe consist of? How old is the universe? Why are we able to exist here on Earth? Some of these may not be answered ever, yet curiosity drives humans to continue researching. 

One of the major components in the universe is the stars. New stars are born from the "ashes" of dead stars and learning how they are formed, evolve and die is fundamental for understanding the environment they evolve in. Such studies are within the field of \textit{Stellar Structure and Evolution}. Describing the different type of stars is an immense task in itself and dividing them into categories and understanding them requires us to understand the processed that leads to differences in their stellar structures. The problem with this is, however, that the insides of stars is unreachable from an observational point of view. We cannot simply look inside the stars, and for a long time penetrating the barriers of stellar layers seemed unobtainable.

Records on observations of pulsating stars goes back nearly 500 years. It was not until 1980's that we fully understood the potential of pulsations. At his point, the field of \textit{asteroseismology} was introduced by \citet{christensen1984} as    
\say{The science of using stellar oscillations for the study of the properties of stars, including their internal structure and dynamics}.

The closest (and therefore most studied) star is the Sun. Knowledge provided from the Sun can be projected to other types of stars, and \textit{helioseismology} (asteroseismology of the Sun) has provided valuable research in the field of Stellar Structure and Evolution. However, helioseismology is not yet developed enough to the reach the deepest layers of the Sun where the core recides. Therefore, we need to turn to other types of stars and exploit their pulsational advantages to reveil the secrets inside the stars. 

Before computers were invented, every observation and calculation were done analytically, which was a very time consuming task. Full stellar evolution tracks could simply ot be carried out. However, we are now at a time where stellar evolution can calculated fully numerically and results compared to the expected parameters derived from observations. Stellar structure and evolution codes can further provide calculations of the pulsations in a star with a stellar pulsation code. This is crucial for stellar modeling as the pulsation frequencies acts as an additional constraint to the parameter space. This constrain can help identify the processes inside pulsating stars, and thereby obtain knowledge on the pulsation mechanism, structure, convection, and evolutionary stage. 

In this work the method of asteroseismic modeling is applied to two different $\delta$ Sct stars, 44 Tau and 187547 (From here on referred to as Superstar). 44 Tau has already been modeled which to estimate the the evolutionary stage. By using a similar method in this work and comparing results, the procedure can be evaluated and adjusted to further be applied to Superstar. Asteroseismic modeling is particularly needed for this star since it represents some of the issues with the theory behind $\delta$ Sct stars. The goal of the asteroseismic modeling in this work is ultimately to analyze how well theory matches observations and using the results to discuss the remaining issues in the field and optimize the methods for any future work.  The work can be utilized on other types of stars, so a deeper understanding of stellar structure and evolution in general can be achieved. Hence, we get one step closer to understanding one of the most fundamental parts of the universe. 

 The work is structured as follows: \chapref{stellarstruc} gives an introduction to Stellar structure and evolution related to  this work. Relevant numerical results are presented to demonstrate key aspects of the field. \chapref{chap:asteroseismology} introduces the theory behind asteroseismology, and how it can be applied to various types of star, depending on the structure of the star. In \chapref{deltascuti} the $delta$ Sct stars described in more detail with the focus being on their importance from an asteroseismic and modeling point of view. Specifically, the asteroseismic and observational background of 44 Tau and Superstar are introduced. The tools used for modeling these star are presented in \chapref{compute}, where the Stellar Structure and Evolution code \texttt{MESA} and Stellar Pulsation code \texttt{GYRE} and their main numerical aspects are presented. \chapref{modeling} described the entire process of modeling the stars and the methods applied to compare to observations. Main results and proposed future work is then discussed in \secref{sec:discussion} and final conclusions are given in \chapref{sec:conclusion}.

This work has made use of data from the European Space Agency (ESA) mission
Gaia (\url{https://www.cosmos.esa.int/gaia}), processed by the Gaia Data Processing and Analysis Consortium (DPAC,
\url{https://www.cosmos.esa.int/web/gaia/dpac/consortium}). Funding for the DPAC
has been provided by national institutions, in particular the institutions
participating in the Gaia Multilateral Agreement.