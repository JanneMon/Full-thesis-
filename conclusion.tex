\chapter{Conclusion}
\label{sec:conclusion}

The goal of this project was to use asteroseismic modeling on the two stars 44 Tau and HD 187547 respectively, in order to extract information on their frequencies, stellar parameters and evolutionary stages. For this purpose, a grid of models was calculated with \texttt{MESA} stellar structure and evolution code to simulate their evolutionary tracks. The \texttt{GYRE}  pulsation code calculated theoretical frequencies for the entire grid to simulate the pulsations within the star. The resolution of the tracks were tested and improven to ensure that all stages of the evolution were properly resolved. 

A \chis method was implemented for both stars, in order to fit observed parameters to modeled parameters.  The \chis was implemented as defined by \eqref{sigma}. The uncertainties of the frequencies were artificially enlarged through the resolution in a second \chis implementation for 44 Tau. The purpose of this was to test if the frequencies needed to be weighed less to compensate for the small uncertainties. In both bases, 44 tau was successfully modeled to be on the post-ms. For the classical \chis approach, the best fitted model was on the tracks with mass $M=1.65M_\odot$....... and the best model is on the post-ms, specifically on the Heney hook, shortly before hydrogen exhaustion. For the artificially enhanced case, the resulting best model was better fitted to the observational parameters \teff,\l and \logg with the evolutionary stage being slightly later. However, the matched modes of the frequencies from models differs from the mode identification more than in the classical case. This indicates that there is a trend running counter between model frequencies and observational parameters. It is not clear whether this is due inadequacy of the models (lack of time-dependent convection treatment and non-adiabatic calculations of the frequencies.), or a bias in the grid stemming from the low $\alpha_{mlt}$ values. 
The results of the evolutionary stage confirms the work from \citet{lenz2010delta}, although with a slightly lower mass. Further investigation is needed in order to address the difference between model parameters of this work and \citet{lenz2010delta}. The grid needs to be narrowed down in parameter space, bout have more grid points instead. This should be done around the best model, in order to test if the \chis can be optimized more options for parameter combinations. The low values of $\alpha_{mlt}$ causes a bias in the models as the frequencies are strongly related to the models. Therefore, the grid needs to be tested with standard values around $\alpha_{mlt}=1.8$ also. 

For HD 187547 the modeling was conducted through fitting the theoretically calculated large frequency separation to two different observed estimates \citet{antoci2011excitation} (private communication,Victoria Antoci, Bedding et al. in review). The modeling was limited to $l=0$ modes with a frequency range of $45-80 d^{-1}$ since modeling the larger frequency separations for higher $l$ requires a more advanced routine than implemented here. 
the results showed that....  
The asymptotic relation is valid when the frequencies decrease linearly with evolution, and this range should therefore be evaluated to exclude models where the frequency range does not behave asymptotically. The results will be biased for young models since the radial order can still be very small even for a high frequency. For future work, the models should therefore be constrained to high-order radial modes where the asymptotic relation is valid. 

Additionally, the high order $l$ should be considered in the calculation of the separation. This could be done by cross-correlation between frequencies for selected model, as this would work for any $l$. 


