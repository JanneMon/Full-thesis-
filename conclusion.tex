\chapter{Conclusion}
\label{sec:conclusion}

The goal of this project was to use asteroseismic modeling on the two stars 44 Tau and HD 187547 respectively, in order to extract information on their frequencies, stellar parameters and evolutionary stages. For this purpose, a grid of models was calculated with \texttt{MESA} stellar structure and evolution code to simulate their evolutionary tracks. The \texttt{GYRE}  pulsation code calculated theoretical frequencies for the entire grid to simulate the pulsations within the star. The resolution of the tracks were tested and improven to ensure that all stages of the evolution were properly resolved. 

A \chis method was implemented for both stars, in order to fit observed parameters to modeled parameters.  The \chis was implemented as defined by \eqref{standard_chi}. The uncertainties of the frequencies were artificially enhanced through the resolution in a second \chis implementation for 44 Tau. The purpose of this was to test if the frequencies needed to be weighed less to compensate for the small uncertainties. In both bases, 44 tau was successfully modeled to be on the post-ms. For the classical \chis approach, the best fitted model was on the tracks with mass $M=1.55$\msun, $X=0.65$, $Z=0.01$,$\alpha_{mlt}$, $\alpha_{ov}$ and the minimum \chis model was shown to be on the post-MS,  well into the post-MS expansion phase. For the artificially enhanced case, the resulting best model was better fitted to the observational parameters \teff,\lum and \logg with the evolutionary stage being slightly later. The results of the evolutionary stage agrees somewhat with the work from \citet{lenz2010delta}(post-MS), although with a lower mass and at an even later stage. The fitted frequencies for the best models showed a discrepancy between $l$ from model and $l$ from mode identification, which is due to the fitting routine used to calculate the \chis not taking $l$ into consideration. An additional result showed that the track with $M=1.65$\msun, $X=0.70$, $Z=0.01$,$\alpha_{mlt}= 0.5$, $\alpha_{ov}=0.3$ has a much better frequency fit (looking at the mode identification also) and an evolutionary stage on the Henyey hook, closer to the results from \citet{lenz2010delta}. However, the mass is underestimated compared to \citet{lenz2010delta} and it is not clear whether this is due to inadequacy of the models (lack of time-dependent convection treatment and non-adiabatic calculations of the frequencies.), or a bias in the grid stemming from the low $\alpha_{mlt}$ values. Even though the frequency fit is significantly better, there is still two modes which differs from observations. This can possibly be explained with uncertainties in the model dependent mode identification.  
Further investigation is needed in order to address the difference between model parameters of this work and \citet{lenz2010delta}. The grid needs to be narrowed down in parameter space and have more grid points. This should be done around the best model, in order to test if the \chis can be optimized more options for parameter combinations. The low values of $\alpha_{mlt}$ causes a bias in the models as the frequencies are strongly related to the models. Therefore, the grid needs to be tested with standard values around $\alpha_{mlt}=1.8$ also. 

For HD 187547 the modeling was conducted through fitting the theoretically calculated large frequency separation to two different observed estimates of $\Delta\nu$ \citep{antoci2011excitation}, (private communication,Victoria Antoci, Bedding et al. in review). The modeling was limited to $l=0$ modes with a frequency range of $45-80 d^{-1}$ since modeling the larger frequency separations for higher $l$ requires a more advanced routine than implemented here. This was done for $\Delta\nu = 3.5 d^{-1}$ and $\Delta\nu = 7 d^{-1}$ respectively. Results showed that for both $\Delta\nu$, the star is on the MS (although at a younger stage for the highest $\Delta\nu$). The  $\Delta\nu = 3.5 d^{-1}$ best track was outside the borders of the instability strip, and the best models \chis for observed parameters and large frequency separation did not overlap. This strongly indicates that $\Delta\nu = 3.5 d^{-1}$ is too low to account for the low value of \lum, and that the star is indeed at the earlier stages of the MS. A cutoff was made in the grid to exclude models on the pre-MS. Since it is an exact cutoff,  the removal of MS models on some track was unavoidable. It is therefore a possibility that the star is even younger than modeled here. For future work, a cutoff should be made based on the helium content in the core, to find the MS on each individual track. Resolution is also poorest at the very early stages of the MS, and more tracks and models are needed to minimize the \chis further. Additionally, the high order $l$ should be considered in the calculation of the separation. This could be done by cross-correlation between frequencies for selected model, as this would work for all values of $l$, and add more constraints to the modeling. 
